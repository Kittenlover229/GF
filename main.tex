\documentclass[twocolumn]{article}

\usepackage[utf8]{inputenc}
\usepackage[T2A]{fontenc}
\usepackage[utf8]{inputenc}
\usepackage[russian]{babel}
\usepackage[margin=0.7in]{geometry}
\usepackage{xspace}



\title{Grimdark Future (перевод правил + корректировка для шестиугольной сетки + хоумбрю)}

\newcommand{\dicespan}[2]{\mbox{#1-#2}}
\newcommand{\h}[1]{\textbf{#1}}
\newcommand{\D}[1][6]{D#1\xspace}

\begin{document}

\maketitle

\section{Общие Принципы}
\subsection{Самое Главное Правило}

Как и в любой сложной игре будут ситуации, не предусмотренные правилами или правила неоднозначны. Когда такое происходит, действуйте так, как подсказывает здравый смысл или личные предпочтения.

Если же игроки не могут решить, как должно действовать определённое правило в конкретной ситуации, используйте следующий метод.

Бросьте кубик. Если выпадает \dicespan{1}{3}, то правило действует так, как решил игрок А, если выпадает \dicespan{4}{6}, то решает игрок Б.

\subsection{Масштаб и Размеры}
Правила игры написаны для круглых подставок диаметром 28 мм. Некоторые модели требуют подставки других размеров, рекомендуется придерживаться следующих размеров для соответсвующих моделей:

\begin{itemize}
    \item \h{Пехота}: от 20мм до 40мм
    \item \h{Мотоциклы и Твари}: 25мм X 70мм
    \item \h{Монстры и Тяжёлая Техника}: 60мм
    \item \h{Транспорт}: без подставки
\end{itemize}

\subsection{Параметры Юнита}
К юнитам приложено множество параметров, определяющих кто они и что могут делать.

\begin{itemize}
    \item \h{Название [Размер]}: Название юнита и количество моделей.
    \item \h{Стойкость}: Бросок для проверок атаки и морали.
    \item \h{Защита}: Бросок для проверок защиты.
    \item \h{Экипировка}: Оружие, которыми обладает этот юнит.
    \item \h{Особые Правила}: Правила, которые применяются к этому юниту.
    \item \h{Стоимость}: Стоимость этого юнита в поинтах.
\end{itemize}

\subsection{Кубы и Броски}
Для игры требуется несколько шестигранных кубов, далее \D[6]. В зависимости от количества моделей на столе рекомендуется иметь от 10 до 20 кубов, также предлагается иметь кубы разной расцветки для ускорения игрового процесса, когда юнит одновременно совершает броски разных типов.

Иногда правила требуют разные кубы, например \D[3], 2\D и \D+1. Для понимания этой формы записи просто пользуйтесь следующими правилами:

\begin{itemize}
    \item \h{\D[3]}: Чтобы узнать значение используйте \D, результат разделите пополам и округлите вверх (напр. 6 на \D  это 3 на \D[3], а 3 на \D это 2 на \D[3])
    \item \h{2\D}: Чтобы узнать значение бросьте два куба и проссумируйте значения на каждом.
    \item \h{\D[6]+1}: Чтобы узнать значение, бросто бросьте кубик и прибавьте к результату единицу.
\end{itemize}

\subsection{Переброс (англ. Re-Roll)}
Когда правило предлагает вам перебросить куб, вы повторно совершаете бросок. Выпавший результат считается финальным, даже если он хуже предыдущего. \emph{Каждый куб можно перебросить только один раз, независимо от того, сколько правил на него действуют.}

\subsection{Столкновение (англ. Roll-Offs)}
Когда правило требует столкновение, все участные игроки совершают бросок и сравнивают результаты. Игрок с наибольшим считается победителем. В случае ничьи игроки должны повторить столкновение, пока не будет определён однозначный победитель.

\subsection{Проверка Стойкости}
В течение игры иногда будет необходимо провести проверку стойкости, чтобы узнать как юнит справится с атакой противника или поддержкой боевого духа. Если правило требует проверку стойкости, бросьте один куб. Если результат совпадает или превышает параметр стойкости юнита, проверка считается пройденной, иначе - проваленой.

\subsection{Модификаторы}
В течение игры некоторые правила предлагают изменить результат броска, обычно повысить или убавить значение на единицу, но конкретные числа могут различаться. Применяя модификатор увеличьте или уменьшите значение на нужное число. \emph{Независимо от модификаторов 1 всегда считается провалом, а 6 - успехом.}

\subsection{Оружие}
Все оружия в игре делятся на две категории: ближнего и дальнего боя. Оружия дальнего боя обладают параметром дальности, используемым для стрельбы, оружия ближнего - нет. Оружия представлены следуюищим образом:
\begin{center}
    \textit{Название (Дальность, Атаки, Особые Правила)}
\end{center}

\end{document}
